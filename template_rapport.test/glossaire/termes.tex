
% explications :
% - https://en.wikibooks.org/wiki/LaTeX/Glossary

% \newglossaryentry{<label>}{
%	 name = {<nom>},
%	 description = {<description>},
%	 first = {<premier appel>},
%	 plural = {<si on ajoute pas de 's' à la fin>},
%	 symbol = {<symbole pi>},
% 	 see = {<label_a_voir},
% }

% \newacronym{<label>}{<abbreviation>}{<full>}
% \newacronym[longplural={pluriel}]{<label>}{<abbreviation>}{<full>}


% utilisation (pour tout):
% - \gls(<label>)
% - \Gls(<label>)  		=> avec majuscule
% - \glspl(<label>)  	=> pluriel
% - \Glspl(<label>)		=> pluriel avec majuscule
% - \glsdesc{<label>}  	=> affiche la description
% - \glssymbol{<label>}	=> affiche le symbole

% utilisation en plus pour les accronymes : 
% - \acrshort{<label>}	=> donne l'abbréviation
% - \acrlong{<label>}	=> donne la description "full"
% - \acrfull{<label>}	=> donne "full (abbreviation)"



\newglossaryentry{mot}{
	name=mot,
	description={un mot},
	first={le premier mot (mot))}
}
