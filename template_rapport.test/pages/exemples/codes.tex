\section{Codes}


\subsection{Lstlisting}


\lstset{language=Haskell}   % Set your language (you can change the language for each code-block optionally)

% frame=single, 
\begin{lstlisting}[language={bash}, caption={}, label={lst:}, float, floatplacement=H]  % Start your code-block

(<*>) :: (Bits a, Num a) => Int -> a -> a             -- produit externe
(<*>) x a
        | x == 1 = a
        | x == 2 = x2 a
        | x == 3 = xor (x2 a) a
    where x2 i
            | testBit i 7 = xor ((shift a 1) - 256) (27)
            | otherwise = shift a 1

mixCol :: [[Int]] -> [[Int]] -> [[Int]]        -- produit matriciel
mixCol const var = [[ foldl xor 0 $ zipWith (<*>) j i | j <- const] | i <- (transpose var)]

\end{lstlisting}



% 2 listings sur la même ligne (avec minipage)
% voir : https://tex.stackexchange.com/questions/35155/lstlisting-in-two-columns


%\lstlistoflistings




\subsection{Minted}

\begin{minted}{java}
/* Java */
package exercice4_Strategie;
public class StrategieIterative extends Strategie {

	protected StrategieIterative test;
	
	@Override
	public static final String inverser(String chaine) {
		StringBuffer strigbuffer = new StringBuffer();
		int test = 12;	// test 
		if(true) return "eee";
		for(int i = 0; i < chaine.length(); i++) {
			strigbuffer.append( chaine.charAt(chaine.length()-1-i) );
		}
		return strigbuffer.toString();
	}
}
\end{minted}


\begin{minted}{cpp}
// C++
#include <iostream> // pour _strdup

using namespace std;	// pour utiliser cout, free, ...

unsigned int Chaine::getSize() {
	// return this -> size;
	return size;
}

const char* Chaine::getString() {
	// return this -> string;
	return string;
}
\end{minted}


\begin{minted}{python}
# Python 

# permet d'afficher une matrice ligne par ligne
def afficher_matrice( m ):
    for ligne in m:
        print(ligne)
    print()

# initialise une matice
def initialiser_matrice( nb_lignes, nb_colonnes ):
    # matrice = [ [ 0 for _ in range(nb_colonnes) ] for _ in range(nb_lignes) ]
    matrice = []
    for i in range( nb_lignes ):
        ligne = []
        for j in range( nb_colonnes ):
            ligne.append(0)
        matrice.append( ligne )
    return matrice
\end{minted}




\begin{minted}{haskell}
-- Haskell
reduitL :: (Num a, Eq a) => [a] -> [a]
reduitL  liste = zipWith (-) (L.tail liste) liste
			
reduit :: (Num a, Eq a) => [a] -> [[a]]
reduit (t:[]) = []								
reduit liste = [ (reduitL liste) ] ++ (reduit (reduitL liste))

diffNewton :: (Num a, Eq a) => [a] -> [a]
diffNewton li = [L.head li] ++ [ (L.head x) | x <- (reduit li)]

vecNewton :: (Eq a, Fractional a, Enum a) => a -> a -> a
vecNewton x nb_fois = foldl (*) 1 [ (x - i) / i | i<-[1..nb_fois] ]
\end{minted}



\begin{minted}{ocaml}
(* OCaml *)

(*ajouter un element a un ensemble*)
let  ajouterEl (el, ens) =
  if appartient(el,ens) then ens
  else Ens(el,ens);;


(*l'union de deux ensemble*)
let rec union (ens1,ens2) =
  match ens1 with
  |Vide -> ens2
  |Ens(t,q) -> union(q,ajouterEl(t,ens2));;

union(e,e2);;
\end{minted}


\begin{minted}{prolog}
/* Prolog */
pow(0, 1).
pow(Puissance, Res) :- 
	Puissance > 0,
	Puissance1 is Puissance - 1,
	pow(Puissance1, Res1),
	Res is 2 * Res1.
% pow(4, R).  => 2*2*2*2=16
\end{minted}


\begin{minted}{bash}
sudo su 
apt-get install update 
apt-get install notepad 
mkdir test
rf -rf ./test/*
\end{minted}


\begin{minted}{tex}
% Latex
\usepackage{tikz}
\usetikzlibrary{automata,fit,trees,matrix,mindmap}
\newcommand{\myunit}{1.1cm}
\usepackage{tkz-graph}
\usepackage{circuitikz}
\section{test}
ceci est un très long texte
\subsection{sous sous}
\end{minted}



\begin{minted}{html}
<!DOCTYPE html>
<html>
    <head>
        <title>DashEB</title>
        <meta http-equiv="content-type" content="text/html; charset=utf-8" />
        <meta name="author" content="Maxime PINEAU">

        <script type="text/javascript" charset="utf-8" src="dist/jquery.js"></script>
        <!-- <script type="text/javascript" charset="utf-8" src="dist/dash.all.js"></script> -->
        <script type="text/javascript" charset="utf-8" src="dist/dash.debug.js"></script>
        <script type="text/javascript" charset="utf-8" src="dist/DashEB.js"></script>
    </head>
    
    <body class="bg">

        <div class="video">
            <video id="player-wrapper" controls></video>
        </div>

        <div>
            <div id="status"></div>
            <div id="state"></div>
            <div id="bufferLength"></div>
            <div id="chunksFromCDN"></div>
            <div id="chunksFromP2P"></div>
        </div>

        <script>
            	var player = new MediaPlayer(context);

                player.startup();
                player.attachView(baliseVideo);
                player.attachSource(url);

                setInterval(updateStats.bind(undefined, player), 500); // maj de l'affichage des metrics
        </script>

    </body>
</html>
\end{minted}


\begin{minted}{js}
var url;
//url = "http://51.255.41.158:8081/test/mystream/manifest.mpd"; 
//url = "http://dash.edgesuite.net/envivio/Envivio-dash2/manifest.mpd";
url = "https://hw.cdn.afrostream.net/vod/24hourlove_TRL/c6832cf78025bcbb.ism/c6832cf78025bcbb.mpd"; // vod 
//url = "https://origin.cdn.afrostream.net/live/bet.isml/bet.mpd";    // live, chaine BET*
//url = "http://rdmedia.bbc.co.uk/dash/ondemand/testcard/1/client_manifest-events.mpd";   // vod, BBC adaptive bitrate test 

var baliseVideo = document.querySelector("#player-wrapper");

//var context = new Dash.di.DashContext();
var context = new DashEB.Context({
	logging: true,
	source: url
});

var player = new MediaPlayer(context);

player.startup();
player.attachView(baliseVideo);
player.attachSource(url);

setInterval(updateStats.bind(undefined, player), 500); // maj de l'affichage des metrics
\end{minted}


\begin{minted}{sql}
/* SQL */
select nomserv, nomproj, nomempl
from basetd.service service , basetd.concerne concerne, basetd.projet projet, basetd.employe employe
where concerne.nuproj = projet.nuproj
and concerne.nuserv = service.nuserv
and projet.resp = employe.nuempl;
\end{minted}


\begin{minted}{php}
<?php
header("Content-type: text/html; charset=utf-8");
require_once("menu.php");

   $menu = affiche_menu_apres_connexion();
?>
 <html>
<head>
    <link href="menu.css" type="text/css" rel="stylesheet" /> 
</head>
<body>    
  
<?php
echo $menu;
?>
\end{minted}



\subsection{test}


\begin{lstlisting}[language=xml, frame=single, caption={"Segment of the XML input file"}, label={lst:xmlfile}]
    some code
\end{lstlisting}


\begin{listing}[H]
    \begin{minted}{c++}
        some code
    \end{minted}
    \caption{"Conversion functions for DirID and ID"}
    \label{lst:conversion}
\end{listing}


\begin{lstlisting}[language=xml, frame=single, caption={"Segment of the XML input file"}, label={lst:xmlfile2}]
    some code
\end{lstlisting}


\begin{listing}[H]
    \begin{minted}{c++}
        some code
    \end{minted}
    \caption{"Conversion functions for DirID and ID"}
    \label{lst:conversion2}
\end{listing}


% minted 
%\listoflistings

% lstlistings
%\lstlistoflistings % obsolète 


