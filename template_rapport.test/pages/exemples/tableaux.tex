
\section{Les tableaux}


\subsection{tableau normal}

% exemple d'un tableau simple
\begin{table}[H]
	\centering
	\caption{Légende du tableau}	
	\label{tab:label_tableau}
	\begin{tabular}{|l|l|l|}
		% header
		\hline \textbf{colonne 1} & \textbf{colonne 2} & \textbf{colonne 3} \\
		
		% le contenu
		\hline texte & texte & texte \\
		\hline 
	\end{tabular}
\end{table}


\subsection{tableau avec du texte long}

\begin{table}[H]
	\centering
	\caption{tableau avec du texte long}	
	\label{tab:label_tableau_texte_long}
	\begin{tabular}{|p{12cm}|p{3cm}|}
		% header
		\hline \textbf{colonne 1} & \textbf{colonne 2} \tabularnewline
		
		% le contenu
		\hline 
			\raggedright \lipsum[1] & 
			\centering texte 
			\tabularnewline
		\hline 
	\end{tabular}
\end{table}


\subsection{tableau ayant beaucoup de lignes}


% pour utiliser les listes dans le tableau
% \setlist[itemize]{label=$-$,leftmargin=*,parsep=0cm,itemsep=0cm,topsep=0cm}


% tableau de 2 colonnes, avec texte centré	
% longtable pour les long tableaux sur plusieurs pages
%\setlength\LTleft{-1.5cm}		% décaler le tableau sur la gauche
\begin{longtable}{|c|c|}
	
	% entête de la première page
	\hline \textbf{colonne 1} & \textbf{colonne 2} \\ 
	\hline
	\endfirsthead
	
	%\hline
	%\endfirstfoot 	
	
	% Entête de toutes les pages	
	%\hline
	%\endhead
	
	% Bas de toutes les pages
	%\hline
	%\endfoot		
	
	% Contenu du tableau 
	\hline 	du texte &	\\ 
	\hline

\end{longtable}



\subsection{tabularx}


\begin{table}[H]
	\centering
	\caption{tabularx}	
	\label{tab:tabularx}
	\begin{tabularx}{\textwidth}{lX}
\hline 
Force    & Force  is a vector  quantity  defined  as the  rate of  change  of
the  momentum  of the  body  that  would  be  induced  by that force  acting  alone . \\ 
\hline 
Moment  of a force   & Moment  of a force  with  respect  to an  origin  is
defined  as the  cross  product  of the  position  vector (with  respect  to
the  same  origin) and  the force . \\
\hline 
	\end{tabularx}
\end{table}



\subsection{Un beau tableau}


\begin{table}[H]
	\centering
	\caption{beautiful table}	
	\label{tab:beautiful}
	\begin{tabular}{llr}  
		\toprule
		\multicolumn{2}{c}{Item} \\
		\cmidrule(r){1-2}
		Animal    & Description & Price (\$) \\
		\midrule
		Gnat      & per gram    & 13.65      \\
		          &    each     & 0.01       \\
		Gnu       & stuffed     & 92.50      \\
		Emu       & stuffed     & 33.33      \\
		Armadillo & frozen      & 8.99       \\
		\bottomrule
	\end{tabular}
\end{table}


% avec dcolumn 
% - http://tex.stackexchange.com/questions/141671/combine-column-types-defined-in-dcolumn-with-tabularx 
\begin{table}
	\centering
	\caption{table using dcolumn}
	\begin{tabular}{ld{3}}

		\toprule
		        &  \multicolumn{1}{l}{\centering col A} \\
		\cmidrule(lr){2-2} 
		\midrule  
		   North &      2,228  \\    
		   South &        689.2 \\
		\bottomrule

	\end{tabular}     
\end{table}

